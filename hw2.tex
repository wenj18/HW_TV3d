\documentclass{article}
\usepackage{mathrsfs}
\usepackage{amsmath}
\usepackage{amsthm}
\usepackage{amssymb}
\usepackage{graphicx}
\usepackage{subfigure}
\usepackage{pythonhighlight}
\setlength{\oddsidemargin}{-0.25 in}
\setlength{\evensidemargin}{-0.25 in} \setlength{\topmargin}{-0.25
in} \setlength{\textwidth}{7 in} \setlength{\textheight}{8.5 in}
\setlength{\headsep}{0.25 in} \setlength{\parindent}{0 in}
\setlength{\parskip}{0.1 in}
\begin{document}
\section{Project 1}
\subsection{PDE}
We get a parabolic equation because of the gradient descent flow
\begin{equation}\label{equ:paowu}
  \left\{
  \begin{split}
    &\dfrac{\partial u}{\partial t}=div\left (\dfrac{\nabla u}{\sqrt{u_x'^2+u_y'^2+u_z'^2+\epsilon}}\right)-\lambda^*(u-f) \\
    &\left. \dfrac{\partial u}{\partial \overrightarrow{n}} \right | _{\partial \Omega} =0\\
    &u(x,0)=f
  \end{split}
  \right.
\end{equation}


\subsection{Numerical format}
For convenience of writing, we define
\begin{equation*}
  \begin{array}{c}
      D_x^{\pm}(u_{i,j,k}) \triangleq \pm (u_{i\pm1,j,k}-u_{i,j,k}),\\
   \end{array}
\end{equation*}
We proposed a numerical approximation of $|\nabla u|$
\begin{equation*}
  \begin{array}{c}
    \sqrt{u_x'^2+u_y'^2+u_z'^2+\epsilon} =  \sqrt{D_x ^+ (u_{i,j,k})^2+D_y^+(u_{i,j,k})^2+D_z^+(u_{i,j,k})^2+\varepsilon}\\
  \end{array}
\end{equation*}
And then, we have
\begin{equation}\label{equ:div1}
  div\left (\frac{\nabla u}{\sqrt{u_x'^2+u_y'^2+u_z'^2+\epsilon}}\right) \approx D_x^-\left(\frac{D_x^+(u_{i,j,k}^n)}{|D_h(u_{i,j,k}^n)|}\right)+D_y^-
  \left(\frac{D_y^+(u_{i,j,k}^n)}{|D_h(u_{i,j,k})^n)|}\right) \triangleq z_{i,j,k}^n
\end{equation}
In addition, we discrete time using forward Euler using forward Euler method. Therefore, the numerical format of parabolic equation \eqref{equ:paowu} is obtained as follow
\begin{equation}\label{equ:shuzhi1}
  u_{i,j,k}^{n+1}=u_{i,j,k}^n +\tau z_{i,j,k}^n-\tau \lambda (u_{i,j,k}^n-u_{i,j,k}^0),\ \ \ \text{where }\tau = t^{n+1}-t^n.
\end{equation}
\subsection{Distribution and communication}
Data distribution:
\begin{python}
  if rank == 0:
    # Generate image
    ...
    # Adding Gaussian noise
    ...
    # Data distribution
    nimgsile = nimg[:,:,0:101]
    sendbuf = nimg[:,:,99:200].copy()
    comm.Send(sendbuf, dest=1, tag=11)
    del nimg, sendbuf
  else:
    nimgsile = np.empty([200,200,101],dtype=np.float64)
    comm.Recv(nimgsile, source=0, tag=11)
\end{python}

Communication:
\begin{python}
  for t in range(T):
    if rank ==0 and not t%5:
      print(t, 'of ', T)
    # In-process computation
    J = worker(nimgsile, J, dt, lam)
    # Blocking communication
    if rank == 0:
      # rank 0: send before recv
      sendbuf = J[:,:,n2-2].copy()
      comm.Send(sendbuf, dest=1, tag=100)
      recbuf = np.empty(J[:,:,n2-1].shape, dtype=J[:,:,n2-1].dtype)	
      comm.Recv(recbuf, source=1, tag=110)	
      J[:,:,n2-1] = recbuf
    else:
      # rank 1: recv before send
      recbuf = np.empty(J[:,:,n2-1].shape, dtype=J[:,:,n2-1].dtype)	
      comm.Recv(recbuf, source=0, tag=100)
      J[:,:,0] = recbuf	
      sendbuf = J[:,:,1].copy()	
      comm.Send(sendbuf, dest=0, tag=110)
\end{python}
\begin{figure}[htp]
\begin{minipage}{0.5\linewidth}
\centering
\includegraphics[width=0.9\linewidth]{./img/nimgsile0.png}
\caption{Process 0}
\label{fig:1a}
\end{minipage}%
\begin{minipage}{0.5\linewidth}
\centering
\includegraphics[width=0.9\linewidth]{./img/nimgsile1.png}
\caption{Process 1}
\label{fig:1b}
\end{minipage}
\end{figure}

\subsection{In-process computation improvement}
Original: Matrix operations (faster than for) but increased 5 times RAM.
\begin{python}
def worker(In, J, dt, lam):
  ep = 1e-4
  m,n,l = J.shape
  # Gradient approximates by forward Euler
  DfJx=J[list(range(1,m))+[m-1],:,:]-J
  DfJy=J[:,list(range(1,n))+[n-1],:]-J
  DfJz=J[:,:,list(range(1,l))+[l-1]]-J
  # \varphi(\nabla u)
  TempDJ=(ep+DfJx*DfJx+DfJy*DfJy+DfJz*DfJz)**(1/2)
  DivJx=DfJx/TempDJ
  DivJy=DfJy/TempDJ
  DivJz=DfJz/TempDJ

  Div=bdx(DivJx,m)+bdy(DivJy,n)+bdz(DivJz,l)
  J += dt * Div -dt*lam*(J-In)
  return J
\end{python}

Improvement: Divided into 8 parts along the x direction, each parts have 25+1 or 25+2 layers. RAM is shown in Figure \ref{fig:2}.

\begin{figure}[htp]
  \centering
  % Requires \usepackage{graphicx}
  \includegraphics[width=0.6\textwidth]{./img/RAM.PNG}\\
  \caption{RAM Improvement}\label{fig:2}
\end{figure}

\subsection{Results}
Results collection and presentation:
\begin{python}
  if rank ==0:
    # Process0 collect data and plt
    deimg = np.empty([nx,ny,nz], dtype = np.float64)
    deimg[:,:,0:100] = J[:,:,0:100]
    recbuf = np.empty([200,200,100],np.float64)
    comm.Recv(recbuf, source=1, tag=20)
    deimg[:,:,100:200] = recbuf
    plt.figure()
    plt.imshow(deimg[100,:,:],"gray")
    ...	
  else:
    sendbuf = J[:,:,1:101].copy()
    comm.Send(sendbuf, dest=0, tag=20)
\end{python}
Results is shown Figure \ref{fig:3a}-\ref{fig:3c}. (Code see \textbf{TV3d2.py})
\begin{figure}[htp]
\begin{minipage}{0.3\linewidth}
\centering
\includegraphics[width=0.9\linewidth]{./img/deimg100x.png}
\caption{y-z Section}
\label{fig:3a}
\end{minipage}%
\begin{minipage}{0.3\linewidth}
\centering
\includegraphics[width=0.9\linewidth]{./img/deimg100y.png}
\caption{x-z Section}
\label{fig:3b}
\end{minipage}
\begin{minipage}{0.3\linewidth}
\centering
\includegraphics[width=0.9\linewidth]{./img/deimg100z.png}
\caption{x-y Section}
\label{fig:3c}
\end{minipage}
\end{figure}


\end{document} 